\documentclass{sicnuthesis}

%在sicnuthesis模板中下面都相当于在定义学校那个丑爆的封面

\title{题目}
\author{}
\school{}
\major{}
\class{}
\studentid{}
\tutor{}
\date{}

\begin{document}

\maketitle

\newpage

\begin{abstract}


摘要是对论文内容不加注释和评论的简短陈述,要求扼要说明研究工作的目的、主要内容、研究结果、结论、
科学意义或应用价值等,是一篇具有独立性和完整性的短文。摘要中不宜使用公式、
图表以及非公知公用的符号和术语,不标注引用文献编号。

摘要内容应在200~400字左右,用宋体小四号字书写。摘要内容后空两行书写“关键词”。
毕业论文、毕业设计行与行之间、段落和层次标题以及各段落之间均为1.5倍行距。

\centering{关键词}

关键词1;关键词2;关键词3(关键词是供检索使用的,主题词条应为通用技术词汇,
不得自造关键词。关键词一般为3~8个,宋体小四号字书写,按词条的学科目录分类顺序,
由高至低顺序排列。毕业论文、毕业设计行与行之间、段落和层次标题以及各段落之间均为1.5倍行距。
关键词与关键词之间用“;”隔开)

(备注:毕业论文、毕业设计摘要与关键词采用英、中两种语言书写,英文在前,中文在后)

\end{abstract}

\newpage

\renewcommand{\abstractname}{abstract} % 这里其实只是换掉了abstract的那个标题(摘要/Abstract)

\begin{abstract}

Abstract abstract abstract abstract abstract abstract abstract abstract abstract abstract 
abstract abstract abstract abstract abstract abstract abstract abstract abstract abstract 
abstract abstract abstract abstract abstract abstract abstract abstract abstract abstract 
abstract abstract abstract.(英文摘要内容必须与中文摘要完全对应。
英文摘要采用Times New Roman小四号字书写,毕业论文、毕业设计行与行之间、
段落和层次标题以及各段落之间均为1.5倍行距。)

\centering{keywords}

Key words;key words; key words
(英文关键词内容必须与中文关键词完全对应。英文关键词采用Times New Roman小四号字书写,
毕业论文、毕业设计行与行之间、段落和层次标题以及各段落之间均为1.5倍行距。
关键词与关键词之间用“;”隔开)
\end{abstract}

\newpage

\tableofcontents

(备注:目录按2~3级标题编写“目录”二字使用黑体小二号字居中书写,隔行书写目录内容,
“摘要、Abstract、正文的一级标题、结论、参考文献、附录目录、致谢”采用黑体小四号字书写,
正文其他层次标题均采用宋体小四号字书写;“摘要、Abstract”与“正文”之间隔一行;
正文中的二级标题、三级标题相对于上一级标题均缩进二个空格书写;目录行与行之间均为1.5倍行距;
目录内容多者,正文中的二、三级标题可使用宋体五号字书写,行与行之间可采用单倍行距。
目录修改时单击目录点右键选择更新域,选择更新整个目录。之后,在“Abstract”的页码后回车加一空行。)

\newpage

\section{前言}

正文采用宋体小四号字,毕业论文、毕业设计行与行之间、段落和层次标题以及各段落之间均为1.5倍行距。

毕业论文的前言应综合评述前人工作,说明论文工作的选题目的、背景和意义、国内外文献综述以及论文所要研究的主要内容,对所研究问题的认识,以及提出问题等。前言只是文章的开头,可不写章号,也可不出现“前言”二字。

毕业设计的前言部分应说明设计的目的、意义、范围及应达到的技术要求;简述课题在国内外的发展概况及存

思想;阐述设计应解决的主要问题。

\section{一级标题}

正文采用宋体小四号字,毕业论文、毕业设计行与行之间、段落和层次标题以及各段落之间均为1.5倍行距。

\subsection{二级标题}

正文采用宋体小四号字,毕业论文、毕业设计行与行之间、段落和层次标题以及各段落之间均为1.5倍行距。

具体内容 具体内容 具体内容 具体内容 具体内容 具体内容 具体内容[[[] 主要责任者.文献题名[J].刊名.出版年份,卷号(期号):起止页码.]]。

\subsubsection{三级标题}

正文采用宋体小四号字,毕业论文、毕业设计行与行之间、段落和层次标题以及各段落之间均为1.5倍行距。
关于文章中公式的具体要求如下。

\begin{equation}
e^x = 1 + \frac{x}{1!} + \frac{x^2}{2!} + \frac{x^3}{3!} + \dots, \quad II < x < I
\end{equation}

正文采用宋体小四号字,毕业论文、毕业设计行与行之间、段落和层次标题以及各段落之间均为1.5倍行距。

具体内容 具体内容 具体内容 具体内容 具体内容 具体内容 具体内容

正文采用宋体小四号字,毕业论文、毕业设计行与行之间、段落和层次标题以及各段落之间均为1.5倍行距。

\end{document}